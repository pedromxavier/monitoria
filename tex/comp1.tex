\documentclass[12pt]{article}

\usepackage{comptype}

\newcommand{\copias}[2]{
\begingroup
	\newcount\foo
	\foo=#2
	\loop
	  #1
	  \advance \foo -1
	\ifnum \foo>0
	\repeat
\endgroup
}

\newcommand{\problem}[2][1]{
	\subsection{#2
		\copias{$\star$}{#1}
		}
}

\newcommand{\chap}[1]{
\pagebreak
\section{#1}
}

\title{Lista de Computação I (Python) \\ {\normalsize DCC / UFRJ}}
\author{Pedro Maciel Xavier (monitor)\\ \texttt{pedromxavier@poli.ufrj.br}}
\begin{document}
	\maketitle
	\section*{Introdução}
	%--- ''Fala galerinha do YouTube!'' --- EXTREME, Venom.
	%~\\
	%~\\
	Essa lista de exercícios ainda se encontra em desenvolvimento. A intenção é que ela tenha um gabarito bem aberto, deixando muito das respostas para a criatividade do aluno. As questões são, em geral, grandes para se resolver e podem necessitar de alguma pesquisa adicional. Elas tem estrelinhas $\star$ indicando a dificuldade estimada. Alguns exercícios foram inspirados em outros propostos em materiais cujas fontes estão devidamente referenciadas no final. É importante você tire suas dúvidas e dê um retorno do que achou dos exercícios através do \texttt{e-mail} no cabeçalho.
	~\\
	~\\
	Boa diversão! \par
	\pagebreak
	
	
	\tableofcontents
	
	\chap{Números (\type{int}, \type{float}, \type{complex})}
	
	\problem[1]{}
	
	\chap{Condições (\stmt{if}, \stmt{elif}, \stmt{else})}
	
	\chap{\textit{Strings} e Texto (\type{str})}

	\problem[3]{DNA}
	
		Uma sequência de DNA (\emph{ácido desoxirribonucleico}) é composta por 4 nucleotídeos: \textbf{A}denina, \textbf{C}itosina, \textbf{G}uanina \textbf{T}imina. No RNA mensageiro (mRNA), a \textbf{T}imina está ausente, e dá lugar para a \textbf{U}racila. Quando um ribossomo realiza a transcrição de DNA em mRNA ele segue uma regra muito simples:
		
		\begin{figure}[H]
		\begin{center}
			\large		
			\begin{tabular}{ccc}
				A & $\rightarrow$ & U\\
				T & $\rightarrow$ & A\\
				C & $\rightarrow$ & G\\
				G & $\rightarrow$ & C\\
			\end{tabular}
		\end{center}
		\end{figure}
	
	\chap{Listas e Tuplas (\type{list}, \type{tuple})}
	
	\problem[2]{Eratóstenes de Cirene}
	
	\chap{Laços (\stmt{for}, \stmt{while})}
	
	\chap{Conjuntos e Dicionários (\type{set}, \type{dict})}
	
	\problem[2]{Mensagem cifrada}	
	
	
	\chap{Arquivos (\stmt{with}, \type{open})}
	
	
\end{document}