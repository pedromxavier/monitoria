\documentclass[12pt]{article}

\usepackage{comptype}



\title{Lista de Computação I (Python) \\ {\normalsize DCC / UFRJ}}
\author{Pedro Maciel Xavier (monitor)\\ \texttt{pedromxavier@poli.ufrj.br}}
\begin{document}
	\maketitle
	\section*{Introdução}
	%--- ''Fala galerinha do YouTube!'' --- EXTREME, Venom.
	%~\\
	%~\\
	Essa lista de exercícios ainda se encontra em desenvolvimento. A intenção é que ela tenha um gabarito bem aberto, deixando muito das respostas para a criatividade do aluno. As questões são, em geral, grandes para se resolver e podem necessitar de alguma pesquisa adicional. Elas tem estrelinhas $\star$ indicando a dificuldade estimada. Alguns exercícios foram inspirados em outros propostos em materiais cujas fontes estão devidamente referenciadas no final. É importante você tire suas dúvidas e dê um retorno do que achou dos exercícios através do \texttt{e-mail} no cabeçalho.
	~\\
	~\\
	Boa diversão! \par
	\pagebreak
	
	
	\tableofcontents
	
	\chap{Condições (\stmt{if}, \stmt{elif}, \stmt{else})}
	
	\chap{Números (\type{int}, \type{float}, \type{complex})}
	
	\problem[3]{Gerando números "aleatórios"}
	
	Gerar um numero aleatório é um pouco complicado em geral. Muitas formas de se fazer isso hoje em dia se baseiam em processos da natureza como fluidos em regime turbulento ou fenômenos quânticos. No entanto, é possível chegar perto disso com muito menos. Para começar, escolhemos uma potência de $2$, como $N = 2^{16} = 65536$. Depois disso, escolhemos um número primo no meio do caminho, digamos, $a = 25773$. Por fim, um númeror menor que o primo e que não seja potência de $2$, por exemplo, $b = 13849$.\\
	\\
	Partimos de um número $n_0$ qualquer no intervalo $[0, 25536)$ e calculamos o sucessor da seguinte forma:
	\begin{align*}
		\mathbf{n}_j = \mathbf{a} \times \mathbf{n}_{j-1} + \mathbf{b} \mod \mathbf{N}
	\end{align*}
	Para obter um número "aleatório" distribuído de maneira uniforme no intervalo $[0, 1)$, basta calcular
		$$\mathbf{x}_j = \frac{\mathbf{n}_j}{\mathbf{N}}$$
	\textbf{Desafio:} Faça o seu próprio gerador de números aleatórios!
	
	\chap{\textit{Strings} e Texto (\type{str})}

	\problem[1]{DNA}
	
		Uma sequência de DNA (\emph{ácido desoxirribonucleico}) é composta por 4 nucleotídeos: \textbf{A}denina, \textbf{C}itosina, \textbf{G}uanina \textbf{T}imina. No RNA mensageiro (mRNA), a \textbf{T}imina está ausente, e dá lugar para a \textbf{U}racila. Quando um ribossomo realiza a transcrição de DNA em mRNA ele segue uma regra muito simples:
		
		\begin{figure}[H]
			\begin{center}
				\large		
				\begin{tabular}{ccc}
					A & $\rightarrow$ & U\\
					T & $\rightarrow$ & A\\
					C & $\rightarrow$ & G\\
					G & $\rightarrow$ & C\\
				\end{tabular}
			\end{center}
		\end{figure}
	Faça uma função que traduza uma fita de DNA (\type{str}) como faria um ribossomo.
	
	\problem[2]{}
	
	\chap{Listas e Tuplas (\type{list}, \type{tuple})}
	
	\problem[2]{Eratóstenes de Cirene}
	
	\chap{Recursividade}	
	
	\chap{Laços (\stmt{for}, \stmt{while})}
	
	\problem[2]{Sequência de Collatz}
	
	A sequência de Collatz é descrita recursivamente por:
	{\large
	\begin{align*}
		f(n) \triangleq \begin{cases}
		3n + 1, &\text{ se } n \text{ for ímpar}\\
		n \div 2, &\text{ se } n \text{ for par}
		\end{cases}
	\end{align*}
	}
	Por exemplo, começamos com $n = 26$. Após sucessivas aplicações temos:
		$$26 \to 13 \to 40 \to 20 \to 10 \to 5 \to 16 \to 8 \to 4 \to 2 \to 1$$
	Isso nos dá uma sequência com $11$ números. $40$ é maior que $26$, mas sua sequência só teria $9$ números.\\
	\\
	Ainda não se sabe se todos os números induzem uma sequência que termina em $1$. No entanto, até agora não foi encontrado um número sequer em que isso não tenha acontecido!\\
	\\
	\textbf{Deasfio: } Faça uma função que calcule o comprimento da sequência gerada a partir de um número natural $n$ qualquer.
	\chap{Conjuntos e Dicionários (\type{set}, \type{dict})}
	
	\problem[2]{Mensagem cifrada}	
	
	
	\chap{Arquivos (\stmt{with}, \type{open})}
	
	
\end{document}